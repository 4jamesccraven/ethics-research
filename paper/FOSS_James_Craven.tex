\documentclass[manuscript,screen,nonacm,12pt]{acmart}
\usepackage{tipa}

%% Dark Mode for editing only.
% \usepackage{xcolor}
% \pagecolor{black}
% \color{white}

\geometry{twoside=false}

\begin{document}

\title{Free and Open Source Software: History, Philosophy, and Impact}

\author{James Craven}
\email{cravenj6@winthrop.edu}
\affiliation{
    \institution{Winthrop University}
    \city{Rock Hill}
    \state{South Carolina}
    \country{USA}
}

\begin{abstract}
{\large \textbf{ABSTRACT}} \\
In the modern age of software development, it is easier than ever to share and
consume software written by anyone, anywhere in the world thanks to the
internet, and hosting services such as GitHub, GitLab, et cetera. In this day
and age it is more important than ever to be educated about the ins and outs of
open source software: its benefits, its dangers, and the precise terms of its
licensing. This paper intends to mitigate these concerns by providing an
overview of the history, philosophy, reality, and legality of FOSS development.
An in-depth discussion of the various licenses available along with their terms
of usage, role in the history of software development, and the reality of how
these projects are managed will provide readers with all that they need to
safely benefit from and contribute to the Free and Open Source Software
Community, both in private, public, and proprietary settings.
\end{abstract}
% \received{20 February 2007}
% \received[revised]{12 March 2009}
% \received[accepted]{5 June 2009}

\maketitle

\section{Introduction}
\subsection{Precise Definition of FOSS}
Free and open source software (FOSS) is a class of software that is distributed
in such a way that the source code of the software is wholly available to the
public (open source) and for which it is permissible to share, modify,
distribute, or sell freely. Strictly speaking, for software to meet the
definition of FOSS, it must follow four tenants of freedom. Users must be
allowed to do the following things with the code without restriction: run,
study, modify, or redistribute it \cite{Fortunato2021}. Notably, this definition
makes no mention of monetisation. It is a common misconception that the "Free"
in FOSS means means free of charge, due to the ambiguous nature of the word
free. The "Free" in FOSS means free in the sense of personal freedoms. Fortunato
and Galassi (2021) reference several common colloquialisms that\textemdash
sometimes humorously\textemdash reference this confusion, such as "think 'free
speech', not 'free beer'" \cite{Fortunato2021}.

\subsection{The Importance of FOSS}
\label{sec:importance}
There are many free and open source software projects that perform a critical
role, both in the user and development space. Among the most famous examples
include the the Linux kernel and the GNU (pronounced \textit{guh-NEW},
\textipa{/g@'n(j)u:/}) system utilities, which collectively make up the
GNU/Linux operating system (typically referred to as just 'Linux'). As of May
2024, 96.3\% of the top 1 million web servers use Linux as their operating
system, approximately 85\% of all smartphones use Android, a Linux distribution,
and Linux's share of the desktop market usage is estimated to be increasing by
800-900 thousand users annually \cite{elad2024}. The vast majority of
programming languages and/or their implementations are free and open source
software. A few of the many examples are C++ (GCC), Python (cpython), Rust, and
Go. Many languages are privately owned by companies initially before open
sourcing, and others have both open and closed source implementations, such as
Java.

Another important class of FOSS are user-space apps that cover a vast array of
applications. For productivity there are entire office suites like LibreOffice,
which is a drop-in replacement for the more well-known, proprietary Microsoft
Office Suite. Image editors like Photoshop have FOSS counterparts in Gimp.
Bourne Shell and its successors such as Bash and Zsh power not only Linux
command lines, but the latter is also the default shell for MacOS, the second
most popular operating system in the world \cite{elad2024}. FOSS is so
widespread and capable that there exists fully functioning end-user operating
systems that have options to completely disallow the usage of proprietary
software, such as Fedora Linux. The importance of FOSS applications in all areas
of software, from systems development, to web development, to producitivity
tools cannot be understated. Without the existence of FOSS, it is highly
improbable that any significant proprietary software could exist, be it due to a
tool it relies on, a library in its codebase, or even the programming language
that it is written in.

This paper will begin with a overview of the history of open source software,
and its development from a hobbyist's pastime to its current status as a
movement and a practice. From there, we will move into a discussion of the
philosophy of the FOSS movement, as well as divisions and the ever-present
freedom-versus-practicality debate. After that overview, we will analyse how
FOSS is carried out in practice, discussing first licensing schemes, then
overall models for managing FOSS projects. A brief discussion of the importance
of this topic will conclude the paper.

\section{The History of Free, Then Open Source, Software}
\subsection{The Early Days, and UNIX}
\label{sec:hist-early}
As discussed in Fortunato and Galassi (2021) \cite{Fortunato2021}, general
consensus on the genesis of the Free and Open Source Software Movement is that
it began with the "hacker" culture of the 1970s. In the early days of computing,
software was tied to the hardware with which it came, due to a lack of
widespread architecture standards. Software was not beholden to intellectual
property rights, as it was tightly "bundled" with the computer itself, and
hobbyists frequently modified programs and shared their modifications. This was
beneficial for both consumers and corporations, as the more amateur developers
contributed to their systems, the more support it would receive and thus the
more marketable their systems would become. In 1969, an event known as the
"unbundling" happened, which marked the beginning of the distribution of
non-system-specific (or "unbundled") software.

Fortunato and Galassi \cite{Fortunato2021} continue, explaining that anti-trust
lawsuits levied at International Business Machines Corporation (IBM), the
substantial time investment that software development required, as well as the
rise of high-level languages such as C were the catalyst for this shift. The
development of C coincided and eventually predicated the development of UNIX, an
operating system being developed at Bell Labs, a subsidiary of the American
Telephone and Telegraph Company (AT\&T). Initially, UNIX was only made available
for internal use at Bell Labs, however from the beginning of the early 1970s
Bell Labs began licensing it out to other groups, primarily universities and
other academic institutions. A select few corporations were also permitted
access to the operating system. The company was unable to release only the UNIX
program or market it as a stand-alone product, so they were forced to license it
in this way, with the source code included. Due to its eventual complete port to
C from its original assembly code and the high-level nature of C, it was the
first operating system to be considered truly portable in the sense of being
able to run on multiple devices. Many people used UNIX in this time, and
frequently interacted with developers at Bell Labs to bring improvements and
updates to the operating system, which created a positive feedback loop in its
adoption. Additionally, many variations were created, such as the Berkley
Software Distribution Group's extended UNIX, which was released under a
permissive license that became the foundation of the modern day BSD license. By
the late 1970s UNIX had "thousands of users" \cite{Fortunato2021}.

\subsection{The Birth of the Free Software Movement}
\label{sec:hist-gnu}
The history outlined in Fortunato and Galassi \cite{Fortunato2021} then goes on
to explain that this up-tick in usage prompted a regression of licensing for
UNIX, as AT\&T sought to maintain economic and intellectual property rights to
UNIX, and limit its distribution. Around this time a programmer for the
Artificial Intelligence Lab at the Massachusetts Institute of Technology (MIT),
named Richard Stallman, became frustrated with the situation and began what he called
GNU, an acronym standing for "GNU's Not UNIX." His goal was to create a UNIX-
like operating system that was free for everyone. Stallman began advertising it
as "Free UNIX," and thus the Free Software Movement  was born. GNU would be a
full operating system, including a kernel, something that other free UNIX
variants at the time notably did not have. A kernel is the part of the operating
system that allocates resources, such as memory, and provides access to the
hardware to programs in the user, or non-kernel space. Additionally, GNU was to
have a full suite of user utilities for all general purpose needs. From the
beginning, the movement was equally ideological as it was technical, intending
to provide all the software that a computer would need, entirely free. He wrote
the GNU Manifesto, and he began recruiting for "[c]ontributions of time, money,
programs[,] and equipment." From this the Free Software Foundation was born, and
Stallman promised permanently free software by excluding GNU software from the
public domain.

\subsection{Moving on From UNIX}
Concerns over the licensing issues of UNIX slowly lead users to begin veering
away from the operating system. After extensive legal battles, BSD Group was
able to keep its UNIX variant, and its free licensing allowed for proprietary
distribution to become popular. From this operating system and its derivatives
came the modern OSX (MacOS) and iOS operating systems. The sharing of forks and
other proprietary OS like early MS-DOS effectively wiped out UNIX. In the 1990s,
GNU had completed its set of utilities, but had still yet to complete its
kernel. Around this time then-Helenski-University-student Linus Torvalds began
working on a kernel as a hobby project, and the Linux Kernel was born. It was
eventually licensed under the GNU Public License, and merged with the GNU core-
utils, GCC, et cetera to become GNU/Linux (hereafter referred to as simply
Linux), the most popular open source operating system in the world
\cite{elad2024}. The introduction of Linux to the public sphere drew more people
into the Free Software Movement and it became central to the movements public
image \cite{Fortunato2021}.

\subsection{Open Source}
\label{sec:hist-osi}
\citeauthor{Fortunato2021}'s \citeyear{Fortunato2021} summary concludes as
follows: the Free Software Movement, despite garnering much attention, had
issues with practical corporate adoption. Larger corporate entities were
hesitant to interact with free software due to the aforementioned ambiguity in
its meaning. In the late 90s, the alternative term "Open Source" was derived,
and from it eventually FOSS arose as a pormanteau of the two adjectives. Open
Source itself was trademarked as a certification with a strict definition, and
is currently managed by the Open Source Initiative (OSI).

\section{FilOSSophy: Freedom vs Practicality}
As mentioned in the previous section, FOSS began as "free software," under the
supervision of the Free Software Foundation. It was not until much later that
the rebrand to open source\textemdash or even FOSS\textemdash was set in motion.
The difference between the two is more than a simple rebrand; as they represent
two different approaches ideologically. This section delves deeper into the
philosophy of these two movements: their core beliefs, key differences, and the
debates that have occurred and continue occurring the FOSS community from both
sides.

\subsection{Free Software}
\label{sec:hist-gnu2}
The aforementioned creation of the the Free Software Foundation was formally
outlined in a collection of documents, not least of which is known as the GNU
Manifesto. In his manifesto, Stallman outlines his reasoning for creating free
software, and in it outlines the beginning of what would eventually come to be
known as a copyleft license: "GNU is not in the public domain. Everyone will be
permitted to modify and redistribute GNU, but no distributor will be allowed to
restrict its further redistribution. [\textellipsis] I want to make sure that
all versions of GNU remain free" \cite{stallman1985}. Stallman further outlined
in other essays the precise definition of free software as he saw it, outlining
the "Four Core Freedoms."

Numbered from 0\textendash3 in reference to zero-indexed arrays in programming,
Stallman issued his freedoms. Freedom zero states that a user must be able to
run as they wish, for any purpose. Freedom one affirms the right of the user to
study the functionality of the code underlying the program, and the ability to
change the code as the user sees fits. He notes that a prerequisite to this
right is that the code be available to the user. The third freedom, Freedom two,
asserts that the user must be able to redistribute the source code, to "help
[one's] neighbor." The final freedom is the ability to redistribute
\textit{modified} versions of the source code to others so that everyone can
benefit from the changes. He again notes the necessity of freely available
source code for this to function \cite{stallman_free2002}.

For Stallman, all of these steps were strictly a matter of right versus wrong,
even going so far as to call the matter a "stark moral choice"
\cite{stallman_free2002}. His precise reasons were various, but to him,
proprietary software was a form of division, and he left his place of work,
saying that he would not be happy to look back on his career if he spent years
aiding and abetting these "walls." He also cites having been on the receiving
end of proprietary software, writing in reference to a time while working at
MIT. There, a printer he made use of was subject to a primary license which he
said made it difficult to work with because he could not alter its functionality
\cite{stallman_free2002}. Ultimately, Stallman saw himself as the ideological
leader of a movement, and his ideas indeed garnered attention and popularity
from 1983\textendash 1996 \cite{klang_2005}.

\subsection{Criticisms, and the Open Source Initiative}
As mentioned in Section \ref{sec:hist-osi}, corporate entities were very
hesitant to interact with something advertised as "free," even with the common
speech versus-beer-analogy \cite{klang_2005}. Many in the Free Software
community began to take issue with this for practical reasons. Desiring to
separate themselves from more ideological ideas, Eric Raymond and Bruce Perens
founded the Open Source Initiative (OSI) in 1998 \cite{OSI2025}.

The main ways in which the OSI distinguishes itself from the FSF is in its
communication and education on the ideas surrounding FOSS development. It aims
to market the utility and benefits of open source software as a means to
increase adoption. As part of this effort, and during the launching of the
Initiative, the Open Source Definition (OSD) was created \cite{OSI2025}. The
definition creates a legal definition for, and extends the definition of, the
Four Core Freedoms. This definition is used as "certification mark" for other
open source licenses \cite{OSI2025}. The exact OSD contains ten requirements to
qualify. The first three encompass the Four Core Freedoms. The fourth
requirement allows for restriction on modification of the author's code if and
only if they allow a method of applying a patch to the program, and thus
modifying it. Requirements five and six disallow personal discrimination and
discrimination against purpose of redistribution, respectively. The seventh
requirement disallows removing rights of any kind from those who receive second-
hand distributions of the software. That is, all people who receive the software
must be subject to the same license. Finally, the last three requirements
dictate that the license must not be unique to any technology or product, and
that they not make requirements of the software that is distributed along side
the author's \cite{OSD2025}.

\subsection{The Modern Philosophy of FOSS}
Stallman would reject the ideas and philosophy that the OSI foments, writing
that "[he] prefer[s] the term 'free software' because, once you have heard that
it refers to freedom rather than price, it calls to mind freedom. The word
'open' never refers to freedom" \cite{stallman_free2002}. While this may have
been true at the time of his writing, current trends have shown otherwise.

In the modern FOSS community, the philosophy and terminology of the OSI seems to
have become the dominant ways of conceptualising and referring to FOSS projects.
It is increasingly rare to see a phrase like "ProjectName, the free X."
Ultimately, the term open source or free and open source is generally preferred.
For instance, the first line of the README for Blender, one of the most used 3D
editing program is as follows "Blender is the free and open source 3D creation
suite" \cite{blender2019}. Additionally many libraries for languages include
"open" in the name, which does not appear to be the case for the word "free." To
name a few examples: openssl, a FOSS library for interacting with SSL
certificates; openjdk, the FOSS implementation of the Java runtime; openSSH, the
SSH daemon that comes pre-installed on most operating systems, including
Microsoft Windows; openBSD one of the modern implementations of Berkley Software
Distribution's UNIX fork; et cetera. On the other hand few use free
directly in the name, and most do so under the Spanish word for free as in
freedom, libre. To name some: LibreOffice, the open office suite; LibreWolf, a
very recent FireFox fork; and FreeBSD, another fork of the original BSD.
Admittedly, without a statistical study of open source naming conventions, this
section is conjectural on the author's part; however, the trend seems to lean
far more heavily in favour the neutral, corporate language of the OSI, whereas
Stallman's ideas and terms are currently considered more fringe.

\section{Open Source Licenses}
Having discussed the history and philosophical basis of free and open source
software, we move into the concrete details of how it works as a development
model, beginning with the various licenses that commonly make up the space.
There are two primary types of open source licenses: permissive licenses and
copyleft licenses. Permissive licenses do exactly as the name suggests, they
license the software to the end user in a permissive way, maximising the users
ability to receive the rights outlined in the Four Core Freedoms. A copyleft
license takes it a step further, and seeks to restrict the rights of corporate
entities. These are the licenses that force all redistributions of the software
to remain open source \cite{morin2012}. This type of license is sometimes
referred to as a "poison pill" license as it forces anyone who wishes to use the
work to also take the free software "pill." 

\subsection{Specific Licenses}
There are numerous licenses that fit the Open Source Definition. The following
subset have been selected for review due to their popularity and/or historical
importance. All of these licenses are provided as templates on GitHub when a
license is created, this fact being an additional reason for their inclusion.

\subsubsection{GNU Public License}
The history of the movement surrounding this license has already been discussed
at length, both in sections \ref{sec:hist-gnu} and \ref{sec:hist-gnu2}. As was
also mentioned before, the GNU Public License (GPL) is a copyleft license. It
was the first of its kind, and was specifically designed to prevent proprietary
redistributions of free software. In addition to the practical legal protections
it confers, the license also comes with a note about the ideological intent
behind the license, and it describes the core beliefs of the Free Software
Movement. According to GitHub's usage statistics, the GPL v3 has consistently
been the 3rd most used open source license in the past 4 years
\cite{githubUsage}.

The genesis of the GPL v1 was the combination (and generalisation) of the
license with several programs produced by the GNU Project, the first of which
being GNU Emacs \cite{history}. This first draft of the GPL had some wording
which Stallman and others considered too unclear, so a second version was drafted
to further clarify these points \cite{history}. This version of the license
remained in use from 1991\textendash2007, when the third and current version of
the license was released. This license was again a minimally changed version of
the former, but this time it added additional protections against practices that
were not explicitly prohibited in previous versions, such as tivoization.
According to Stallman, tivoization is a process by which a manufacturer of
computerised appliances allow you to modify the programs running on them, but
shut down if modified software is detected, thus creating a \textit{de facto}
prevention of code modification.

The permissions provided by the license are as follows: commercial use,
modification of the source code, distribution of the source code, patent rights,
and private use. As with most licenses, it limits liability of the author and
denies warranty to the end user. Use of code under the GPL v3 license has the
following stipulations: a copy of the license and the copyright notice must be
included with distributions of the program, modifications must be disclosed and
documented, the source code must be disclosed, and any modifications must also
be released under the same license \cite{GPLv3}.

\subsubsection{Massachusetts Institute of Technology License}
\label{sec:mit-license}
The MIT License, much like the GPL, started as a collection of different
licenses as opposed to a singular document. According to Jerome Saltzer, a
source from MIT, in late 1983, a senior researcher at MIT, named David Clark,
wrote networking software that began to garner wider interest which brought up
questions of how the program should be licensed to users outside of MIT. The
group decided that their primary objective was to influence the way software is
done, and believed that licensing fees would be detrimental to that goal. As
such, the MIT Licenses was born \cite{SaltzerMITLicense}. The researchers' goal
of influencing the software world was undoubtedly realised. A few years after
its initial release, Xorg, the developers of the X Window System for Linux,
distributed their windowing system under a slightly modified MIT license. The
window manager would go on to be the most popular window manager on Linux for
many years, being used on millions of devices \cite{SaltzerMITLicense}. On
GitHub, the license is one of the most popular and well-known open-source
licenses, and many amateur programmers use it to license one-off projects.
According to GitHub's usage statistics, the MIT License has been the number one
most used license since at least 2020 \cite{githubUsage}.

Unlike the GPL, the MIT License is not a copyleft license, but a permissive
license. The license allows for usage of the software for any reason, regardless
of intent. This includes commercial purpose. Like other software licenses, it
disclaims liability and warranty. The only stipulation of the MIT License is
that distributions retain a copy of their license. The simplicity of the license
and its language has been a key factor in its wide adoption, a shortened excerpt
from which may be seen below \cite{MITLicense}:

\begin{quote}
    Permission is hereby granted, free of charge, [...] to deal in the Software
    without restriction, [...] without limitation the rights to use, copy,
    modify, merge, publish, distribute, sublicense, and/or sell copies of the
    Software, [...] subject to the following conditions: The above copyright
    notice and this permission notice shall be included in all copies or
    substantial portions of the Software.
\end{quote}

Excluding the last portion of the license, which includes the aforementioned
liability and warranty disclaimers, the above is essentially the entirety of the
license. It is short, clear, concise, and protects the developer from the
primary dangers of releasing software publicly.

\subsubsection{Apache License}
In 1993, a group of developers began sharing patches to the NCSA Networking
Daemon, creating the beginnings of what would become the Apache Server software.
By 1999, the group incorporated, becoming what is now known as the Apache
Software Foundation. When it came time to release the software, they wanted to
create a license that was open source, but more permissive \cite{apacheHistory}.
Like the MIT License, the Apache License is permissive, not copyleft. It has
been a very influential license, as it has consistently been the second most
used named license on GitHub \cite{githubUsage}.

The license itself affords the following permissions: commercial and private
use, modification, distribution, and patent use. Like the other license, it has
disclaimers for liability and warranty. Unseen thus far, however, is an explicit
disclaimer that the license does not grant trademark rights. It is worth noting
that this is true of other licenses as well; however, the Apache License notes
this explicitly. Like the MIT License, the Apache License stipulates that the
license must be included with distributions of the license. It also requires
that any modifications of any kind be documented prior to redistribution
\cite{ApacheLicense}.

\subsubsection{Berkley Software Distribution License}
The history of the Berkley Software Distribution (BSD) License was already
touched on briefly in Section \ref{sec:hist-early}. When the BSD Group began
distributing their variants of UNIX, initially they could not do so in its
entirety, as some of the code was still from Bell Labs, and thus proprietary
after AT\&T began clamping down on its distribution \cite{Fortunato2021}. To
combat this, the BSD Group distributed individual components of the system
pending the rewrite, and the permissive BSD License was born.

The first official version of the license is known as the BSD 4-Clause license
or the "old license". Alongside the standard disclaimers for liability and
warranty, it has the following four requirements, which give it its name
\cite{BSD4Clause}:

\begin{quote}
    \begin{enumerate}
        \item Redistributions of source code must retain the above copyright
        notice, this list of conditions and the following disclaimer.
        \item Redistributions in binary form must reproduce the above copyright
        notice, this list of conditions and the following disclaimer in the
        documentation and/or other materials provided with the distribution.
        \item All advertising materials mentioning features or use of this
        software must display the following acknowledgement: This product
        includes software developed by the organization.
        \item Neither the name of the copyright holder nor the names the
        copyright holder nor the names of its contributors may be used to
        endorse or promote products derived from this software without specific
        prior written permission.
    \end{enumerate}
\end{quote}

With time, this version of the license became less popular, in part due to some
of the restrictions it imposes. Today the four-clause license is not directly
available as a template on GitHub, but one can find instead the 3- and 2-clause
licenses \cite{githubUsage}. The more popular of the two, which until 2022
ranked as the fourth most popular license and more recently as the fifth
\cite{githubUsage}, is the three-clause variant.

The three clause is largely identical to the "old license," but it omits the
third clause which makes reference to advertising materials \cite{BSD3Clause}.
The other popular variant of this license is the two-clause variant which
additionally removes what was originally the fourth clause, leaving only
restrictions on redistributions \cite{BSD2Clause}. This final variant, sometimes
referred to as the "simple" variant, has ranked 10\textendash12 on GitHub from
2020\textendash2024 \cite{githubUsage}.

\subsubsection{The Unlicense}
Another license that is more recent and fairly popular is the Unlicense. In
2010, Arto Bendiken wrote a fiery blog post denouncing copyleft licenses such as
the GPL, as well as the Free Software Movement as a whole. He likened the use of
such copyright notices as threatening to send "men with guns" after those who
infringe on the license \cite{bendiken_2025}. Citing these reasons, he wrote
The Unlicense.

The Unlicense seeks to do exactly what it implies, by entirely removing all
copyright claim to the work, effectively permitting anyone to do anything with
it. The license file\textemdash which Bendiken recommends to be saved to an
UNLICENSE file instead of the traditional LICENSE file\textemdash says the
following \cite{unlicense}:

\begin{quotation}
    This is free and unencumbered software released into the public domain.

    Anyone is free to copy, modify, publish, use, compile, sell, or
    distribute this software, either in source code form or as a compiled
    binary, for any purpose, commercial or non-commercial, and by any
    means.
    
    In jurisdictions that recognize copyright laws, the author or authors
    of this software dedicate any and all copyright interest in the
    software to the public domain. We make this dedication for the benefit
    of the public at large and to the detriment of our heirs and
    successors. We intend this dedication to be an overt act of
    relinquishment in perpetuity of all present and future rights to this
    software under copyright law. 
\end{quotation}

To better understand Bendiken's reason for this license, it is necessary to
examine his conception of licenses. Up until this point, this paper has
envisioned open source licenses as one of two camps: permissive or copyleft. To
Bendiken, this is a false dichotomy. He argues that a third type of license,
public domain licenses, should also be considered a part of FOSS. He argues that
even permissive license are too restrictive, and that requiring attribution is
"holding a gun to [one's] head" \cite{bendiken_2025}.

The Unlicense, despite its relative recency, is still a popular license, perhaps
in part due to its inclusion as an option in GitHub's licensing templates. At
the height of The Unlicense's popularity, it was found at the eighth position in
GitHub's usage statistics, more recently falling to ninth \cite{githubUsage}.

\subsection{Statistical Limitations}
If conclusions can be drawn from GitHub's statistics alone, it is clear that
most developers seem to prefer a permissive license over a copyleft license like
that of the Free Software Foundation. Nothing can be said concretely about the
reasons \textit{why} this is the case, however. The vast majority of the
licenses covered here are permissive, and that in and of itself may contribute
to the high "preference" for permissive licenses. Also, the sample in question
considers only GitHub which, while being wildly popular, is not the only
place in which FOSS may be distributed. These licenses, in addition to a few
others, are the only ones available in GitHub's template selection, and thus are
more likely to be selected within GitHub.

\section{Governance of FOSS Projects}
\label{sec:gov}
FOSS projects encompass a wide variety of use cases, audiences, and sizes. Given
the wide spread of possibilities, there is no one-size-fits-all solution for
governing these projects. Generally, there are two main types of participants
that are present in all governance styles: a committer and contributor.
Generally speaking, a contributor is anyone who contributes time, resources, or
code to a FOSS project. A committer is a bit less strictly defined, but their
role is typically characterised by direct access to the code base. The precise
authority or lack thereof that a committer has depends on the government model
of the project \cite{Ritvo2017}.

\subsection{Models of Government}
\subsubsection{Benevolent Dictator for Life}
One of the most well-known leadership systems for an open source project is the
Benevolent Dictatorship. This style of government is defined by the existence of
a Benevolent Dictator for Life (usually shortened to BDFL). A BDFL is a single
person\textemdash or rarely, persons\textemdash that has final say over any
decisions related to a project. Despite the name, which is largely used with
humorous intent rather than desire for true association with a real-life
dictatorship, BDFLs tend to be rather permissive in overall project direction,
allowing for day-to-day decisions to be made by committers rather than
micro-managing every detail of the project \cite{Ritvo2017}.

The most famous project that is an example of this governance model is the Linux
kernel, which is dictated by the eponymous Linus Torvalds. The structure of the
project is such that the various components are generally overseen by certain
committers\textemdash which are referred to as maintainers within the Linux
project\textemdash and Linus only steps in for conflict resolution and other
major decisions \cite{Ritvo2017}. Not all BDFLs are actually for life, however.
An example of this is Guido Van Rossum, the Dutch programmer who created the
Python programming language. Unlike Torvalds, Rossum has abdicated his position
as of 2018, and the Python project is no longer run under this model
\cite{fairchild_2018}.

\subsubsection{Meritocracy}
Another style of governance is a meritocracy. Meritocracies are more
egalitarian; all members are considered equals, but it is possible to gain more
influence and general standing with the community and thus the project by
proving one's "merit," that is, contributing time to the project. Overall, it is
the community who makes decisions about what direction a project should take in
a meritocracy, but committers who have attained their status by donating time
have an easier time affecting change. As mention in Section \ref{sec:gov}, a
committer is one who has direct access to the codebase. When a contributor
wishes to add to the project, they submit their patches for consideration, and
the committers deliberate over its addition. Conversely, when a committer wishes
to make a change, it must be agreed upon by the community. Ultimately, the role
of a committer in a meritocracy is one of necessity more so than status
\cite{Ritvo2017}.

\subsubsection{Delegated Governance}
It is often the case that the founder of a project, when disinterested in
becoming a BDFL but still wanting something more rigid than a meritocracy, will
create an official administrative body to govern the projects direction before
relinquishing their temporary dictatorship. In this style of government, a
temporary committee is appointed by the founder of the project, and from there
delegates are appointed through various means common to real life governments,
such as elections or appointments through various outlets. The councils and
committees that head delegated FOSS projects are a distinct, third class of
project member (though there may be overlap between the council and committers,
et cetera) \cite{Ritvo2017}.

\subsubsection{Dynamic Governance}
The final model discussed in this section will be the dynamic governance model.
The dynamic governance model is defined by a wide dispersion of control
throughout the community surrounding the project. Multiple committee-like
structures exist that handle different parts of the project, and there is no
top-down hierarchy across the project as a whole \cite{Ritvo2017}. While they
may seem similar, this model is distinct from a meritocracy. Unlike a
meritocracy, the groups that hold decision making power in a dynamic governance
model is generally an organised group of people with specific, definable
structure.

\subsection{Logistics}
The nature of FOSS makes it paramount that a clearly defined method of
communicating about, contributing to, and reviewing patches exists and is
accessible to all members of the project's community. Traditionally this has
been done with mailing lists and other decentralised methods of similar style.
Nowadays, many newer projects have gravitated to git-based hosting websites such
as GitHub and GitLab. Often feature requests, bug reports, and other topics with
relation to the direction of these projects are hosted on these websites. While
the day-to-day operations such as merging patch or pull requests are handled on
sites like GitHub, often news, blogs, and documentation are hosted on associated
websites as well. These tools are essential to the governance and operation of
open source projects.

% \section{Challenges}
% The merits of open source software have been lauded by many, and the licensing
% model is clearly capable of producing great and at times revolutionary software.
% That is not to say that open source software is without its challenges, however.
% As with any group of humans with a population greater than one, FOSS projects
% often have interpersonal conflicts. FOSS also deals with the same issues that
% plague any software project, such as issues finding cost-effective hosting,
% legal troubles, et cetera.
% 
% \subsection{Case Studies}
% To get an understanding of some of the issues that open source projects face, an
% analysis will be made of three situations that have happened or continued to
% happen within the past few years in notable free and open source software
% projects. Ranking from least to most obscure, the following will be analysed:
% The inclusion of the Rust programming language in the Linux Kernel as well as
% Freedesktop.org's recent issues with hosting for GitLab and other infrastructure
% related to the Freedesktop project. These problems are a mixture of
% interpersonal and technical issues, within multiple governance models. It is the
% hope of the author that these issues, or rather the very different circumstances
% they entail, will provide an overview of these issues better than any summary on
% the author's part could.
% 
% \subsubsection{Rust in the Linux Kernel}
% In observing discourse about the Rust programming language, one thing is
% blatantly obvious: many developers have \textit{strong} opinions about the
% language, and those who have some opinion outnumber those who do not
% significantly. It should come as no surprise that this is also true of those who
% use, contribute to, or profit from the Linux Kernel. The primary method that
% discussion is had about the Linux Kernel is through the Linux Kernel Mailing
% List (LKML).
% 
% The first semi-official mention of Rust in the Linux Kernel comes from an RFC
% authored by Miguel Ojeda in mid-April, 2021 \cite{RustRFC2021}. An
% RFC\textemdash or Request For Comments\textemdash is a type of preliminary
% proposal for additions to a project, often with a draft patch that incorporates
% a minimal viable implementation of the suggested feature. The RFC contained the
% work of many contributors, and outlined only a framework to begin adding Rust
% into the compilation process, not an effort to re-write any core kernel
% functionality. It was abundantly clear from the prose of the RFC that the
% attempt to incorporate Rust was to be gradual rather than immediate, and the
% authors sought constructive criticism in good faith.
% 
% There was a great deal of constructive feedback provided in the replies, mostly
% technical in nature, but immediately there was backlash from long-time kernel
% developers. For example, Peter Zijlstra was very critical of the addition as a
% whole. Most of the concerns outlined by Zijlstra were concerned with the style
% of rust and the Rust-For-Linux Project, rather than technical concerns (though
% he did cite concerns of having two languages from a maintenance perspective).
% 
% The first concern Zijlstra cited was the lack of ability to view the
% documentation for the project in a terminal, stating that there were many
% "freaks" like him who developed for the terminal and desired that capability. He
% said that "HTML is not a valid documentation format," and that Markdown itself
% is "barely readable" \cite{zijlstra_2025}. In later replies he becomes more
% hostile about the changes stating "You're asking to join us, not the other way
% around. I'm fine in a world without Rust."
% 
% This problem is still present even in 2025, with senior members of the kernel
% still having issues with Rust's inclusion, which has been back by Torvalds
% himself since the early days of adoption. In February 2025, tensions came to a
% head once more over to what extent Rust is and is not allowed in the kernel. In
% response to a mail thread clarifying some information about the policies, of the
% Rust-For-Linux Project Christopher Hellwig, then maintainer of the DMA Mapping
% components of the kernel, replied, becoming highly critical of the changes, and
% of Torvalds's alleged lack of clarity on his personal policy towards the
% project. Hellwig claimed that, in public it was stated that Rust code would not
% be forced upon any components of the kernel, but that in private Torvalds had
% said that he would merge any Rust code regardless of maintainer protest
% \cite{hellwig_2025}.
% 
% This particular bout of controversy had been initiated by an instance where
% a patch had been submitted to the kernel, and Hellwig had denied its inclusion
% due to concerns about interoperability of Rust with the C interface for his
% component of the kernel. This, alongside his reply to the email thread, caused
% a great deal of conflict. Ten days after Hellwig's initial reply, Torvalds
% intervened, siding against Hellwig. He stated that Hellwig's argument was
% "garbage," and that it was baseless as the code he denied was not related to
% the DMA layer's code, it only called it \cite{torvalds_2025}. This ultimately
% led to Hellwig stepping down as maintainer of that layer.
% 
% Four years after its initial introduction to the kernel, Rust adoption is still
% ongoing, and is supported by Torvalds, but it is not without its opposition.
% Given the governance model of the Linux Kernel, it is unlikely that Rust will be
% leaving the kernel anytime soon.
% 
% \subsubsection{Freedesktop.org Hosting Troubles}
% Many open source software projects have a problem not present in the corporate
% world, which is the need to find hosting solutions for substantially cheaper
% than they can typically be purchased. Often, projects rely on donations to make
% ends meet, and Freedesktop.org, the creators of the GNOME Desktop Environment
% are not exceptions to this.
% 
% Until recently, the Freedesktop Project, and by extension all projects hosted on
% GitLab, were hosted by a servers owned by the company Equinix. In early
% February, Equinix made the decision to shut down the bare-metal server hosting,
% and because of this they would no longer be able to host Freedesktop et cetera
% \cite{posch_2025}. Events like these have far-reaching effects in the FOSS
% community. In Section \ref{sec:mit-license}, the historical importance of the
% Xorg display server was explained. Currently, public opinion among Linux users
% is slowly shifting towards a new display protocol, called Wayland. Wayland's
% development was hindered by the GitLab shutdown, as it was hosted on the
% git-sharing site.
% 
% As of late February, a solution was found in Hetzner, a German Kubernetes
% server-hosting solution \cite{larabel_2025}. After a period of time the services
% became mostly operation again. As of April 13th, 2025, a banner at the top of
% GitLab.com still has warnings of minor technical issues caused by the migration,
% but is otherwise operational. The banner reads: "The migration is almost done,
% at least the rest should happen in the background. There are still a few
% technical difference between the old cluster and the new ones, and they are
% summarized in [a help page on the website]."
% 
% \subsection{Discussion}
% The case studies outline in this section serve to highlight the issues that come
% with maintaining FOSS projects. A common analogy coined by Eric Stallman, to
% represent the different ways of governing these projects is the Cathedral and
% the Bazaar (which originally comes from an essay of the same name). He likens
% the way Linus Torvalds runs his project to a bazaar, or a market, where people
% come and go freely, buying and selling in a very decentralised way. The style of
% issue that arises from this type of project is evident; when a "merchant" in the
% bazaar, like Peter Zijlstra, is displeased, it can cause a great deal of
% disruption, and conflict resolution becomes a great concern for the Benevolent
% Dictator. Conversely, when Freedesktop's leadership is much more structured and
% centralised. But maintaining a cathedral is no small feat, and relies on the
% tithes of its congregation.

\section{Conclusion}
Since knowledge and experience are paramount to being a desirable candidate in
the software development field, the ability to easily showcase this is crucial.
One of the best ways to do this is by starting or contributing to projects. Open
source is often an accessible way of building this portfolio.

As such, it is essential that junior software developers are familiar with the
ins and outs of open source software development, as well as the applicability
and ethics of the most common licenses that are encountered in the FOSS
ecosystem. This paper serves to provide critical background knowledge, as well
as practical information on these licenses, open source software projects, and
the challenges that lie within.

The usefulness of these concepts are not confined to the FOSS world, however.
When making proprietary business software, it is often emphasised that
programmers should not re-invent the wheel. In the age of the internet, it is
easy to find software to serve just about any purpose. It is another question
entirely to find software that can \textit{legally} fit one's need in a
proprietary setting.

Software engineers of today and tomorrow must be intentional and considerate
with respect to licensing, whether it be licensing of the projects that they
create, sub-licensing a library that their code base relies on, or carefully
ensuring that the legal requirements of redistributed software are met. By
familiarising ourselves with these concepts, we make better, more user-friendly,
and more ethical software.

\newpage

\bibliographystyle{ACM-Reference-Format}
\bibliography{FOSS_James_Craven}

\end{document}
